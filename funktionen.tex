%!TEX root = /Users/stwaidele/Dropbox (Leisinger)/02 - AKAD/Projektbericht/Möglichkeiten der Digitalen Kontaktaufnahme im Endkundenbereich/vorlage.tex

%% Definition for Codeschnipsel im Fließtext
\newcommand{\code}{\texttt}
% \newcommand{\buzz}{\textit}
\newcommand{\buzz}{\textit}

\newcommand{\todo}[1]{\fbox{\parbox{\textwidth}{\textbf{To do:} #1}}}
%\newcommand{\myref}[1]{„\ref{#1}~\nameref{#1}“}
\newcommand{\myref}[1]{\textit{\ref{#1}~\nameref{#1}}}

%% Für Codeblöcke mit Syntax-Highlighting
%% http://www.ctan.org/tex-archive/macros/latex/contrib/minted/
\usepackage{minted}
\definecolor{bg}{rgb}{0.95,0.95,0.95}


%% Für Zeilenumbruch langer Quelltextzeilen
%% http://tex.stackexchange.com/questions/112559/box-around-minted-environment
\usepackage{tcolorbox}
\usepackage{lineno}
\def\gobble#1{}
\renewcommand\DeleteFile[1]{}
\usepackage{xparse}
\ExplSyntaxOn
\box_new:N \l_fvrb_box
\tl_new:N \l_fvrb_tl
\RenewDocumentCommand \FancyVerbFormatLine { m }
 {
   \hbox_set:Nn \l_fvrb_box { #1 }
    \dim_compare:nNnTF { \box_wd:N \l_fvrb_box }>{ \linewidth }
      {%box to big
       \tl_set:Nn \l_fvrb_tl { #1 }
       \fvrb_use_tl:N \l_fvrb_tl
      }
      {%box fits
       \box_use:N \l_fvrb_box
      }
 }
\cs_new:Npn \fvrb_use_tl:N  #1
 {
  \group_begin:
   \null\hfill\vbox_set:Nn \l_fvrb_box
     {\hsize=\linewidth
      \renewcommand\thelinenumber
           {
             \ifnum\value{linenumber}=1\relax\else
                  $\rightarrow$
             \fi
           }
      \begin{internallinenumbers}
        \advance\hsize by -2em
        \hspace*{-2em}\tl_use:N #1
      \end{internallinenumbers}
     }
   \box_use:N \l_fvrb_box
  \group_end:
}
\ExplSyntaxOff
\usepackage{etoolbox}
\BeforeBeginEnvironment{minted}{\begin{tcolorbox}}%
\AfterEndEnvironment{minted}{\end{tcolorbox}}%

% Eigenes Verzeichnis für Listings
\newfloat{quellcode}{tbp}{lop}
\floatname{quellcode}{Quellcode}

%% Feste Spaltenbreite
%% http://de.wikibooks.org/wiki/LaTeX-W%C3%B6rterbuch:_tabular
\usepackage{tabularx}
\newcolumntype{L}[1]{>{\raggedright\arraybackslash}p{#1}} % linksbündig mit Breitenangabe
\newcolumntype{C}[1]{>{\centering\arraybackslash}p{#1}} % zentriert mit Breitenangabe
\newcolumntype{R}[1]{>{\raggedleft\arraybackslash}p{#1}} % rechtsbündig mit Breitenangabe