\addtocontents{toc}{\protect\newpage}
\section{Fazit \& Ausblick} % (fold)
\label{sec:fazit_ausblick}

\subsection{Fazit} % (fold)
\label{sub:fazit}

% subsection fazit (end)

\subsection{Ausblick} % (fold)
\label{sub:ausblick}
Ausbau der Anwendung, um verschiedene Szenarien simulieren zu können, z.B. variables Arbeitstempo der Erzeuger (höhere Belastung von Maschinen oder Menschen), zusätzlichen Erzeugern (Wie lange dauert das Anfahren?), Hinterlegung von Kosten für Lagerhaltung, Wartezeiten bei Erzeuger oder Verbraucher, etc. Bis hin zu hinterlegter Kalkulation, das den Endpreis des Produktes ermittelt und somit steuernd auf die Nachfrage einwirkt.
Hierdurch könnten Aussagen zu Stoßzeiten (Gastronomie, Einzelhandel, Energie, Webserver, ...) getroffen werden. 

Für eine solche feingliedrige Simulation ist eine eigene Implementierung der Queue oder evt. sogar des Threading denkbar, um an beliebigen Stellen den Zustand des Systems abfegen und analysieren zu können.
% subsection ausblick (end)

% section fazit_ausblick (end)