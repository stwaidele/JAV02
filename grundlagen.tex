%!TEX root = /Users/stwaidele/Dropbox (Leisinger)/02 - AKAD/Projektbericht/Möglichkeiten der Digitalen Kontaktaufnahme im Endkundenbereich/vorlage.tex

\section{Erzeuger und Verbraucher} % (fold)
\label{sec:erzeuger_und_verbraucher}

\subsection{Erzeuger–Verbraucher—Muster} % (fold)
\label{sub:erzeuger_verbraucher_muster}

Bei Erzeuger–Verbraucher–Konstellationen treten Produzenten und Konsumenten miteinander über eine Warteschlange bzw. einen Buffer miteinander in Kontakt. Jeder der Akteure handelt hierbei unabhängig von den anderen und quasi gleichzeitig.\footnote{vgl. \cite{osdi}, S. 76f}

Hierbei ist es unerheblich, was hierbei erzeugt bzw. verbraucht wird. Es kann sich um reale Güter handeln, die von Kunden nachgefragt werden, oder auch um Informationen, die Abgerufen werden. Ebenfalls ist es unerheblich, ob der Erzeuger tatsächlich den Bedarf des Verbrauchers befriedigt, oder ob Anfragen erzeugt werden, die dann vom Verbraucher abgearbeitet bzw. gelöst werden.\footnote{Ein Beispiel hierfür ist z.B. ein Trouble–Ticket–System: Hier werden vom Nutzer Supportanfragen „erzeugt“, die dann von den Supportmitarbeitern „verbraucht” werden.} 
Auch können die Anzahl der Erzeuger und die der Verbraucher beliebig gewählt werden oder sogar schwanken. Allen Situationen ist jedoch gemeinsam, dass das Erzeugnis nur einem Verbraucher zur Verfügung steht und nur ein mal konsumiert werden kann.

Durch das einfügen eines Zwischenspeichers stellen Erzeuger–\-Verbraucher–\-Kon\-stel\-lation\-en einen Mechanismus dar, 
durch den eine beliebige Anzahl von Produzenten und Verbraucher miteinander kommunizieren können.
Es handelt es sich somit um ein Entuwrfsmuster, welches in der Programmierung von nebenläufigen Anwendungen.\footnote{vgl. \cite{openmp}, S. 111} 

% subsection erzeuger_verbraucher_muster (end)

\subsection{Zwischenspeicher} % (fold)
\label{sub:buffer}

Bei der Zwischenspeicher (engl. Buffer) können verschiedene Prinzipien zum Einsatz kommen. Bei \ac{LIFO} werden diejenigen Elemente zuerst entnommen, die als letztes hinzugefügt wurden. Dieses Verfahren kommt bei der Implementierung mithilfe eines Kellerspeichers (engl. Stack) zum Tragen.  Das \ac{FIFO} Prinzip, bei dem das zuerst hinzugefügte Element auch als erstes wieder entnommen wird kann durch eine Warteschlange (engl. Queue) oder mithilfe eines Ringbuffers implementiert werden.\footnote{vgl. \cite{algorithms}, Abschnitt 1.3.3.8 und 1.3.3.9}

Bei der im Rahmen dieser Arbeit wird eine FIFO–Warteschlange (engl. Queue) genutzt.\footnote{vgl. \cite{javadoc:lbq}}

% subsection buffer (end)

\subsection{Kritische Abschnitte} % (fold)
\label{sub:kritische_abschnitte}

Damit Verbraucher nicht auf eine leere Queue zugreift, muss zunächst geprüft werden, ob ein Element zur Verfügung steht. Da Prüfung und Zugriff jedoch getrennte Operationen darstellen, besteht die Möglichkeit, dass ein weiterer Verbraucher das letzte Element zwischen diesen beiden Vorgängen entnimmt.\footnote{Bei mehreren Erzeugern tritt das Problem analog bei fast voller Warteschlange auf.} Um solche konkurrierende Zugriffe zu Verhindern, muss verhindert werden, dass sich mehrere Verbraucher gleichzeitig in diesem kritischen Abschnitt befinden.

Eine solche Zugangskontrolle lässt im sich Programm mithilfe von Mutex' bzw. Semaphoren realisieren. Beiden Konzepten ist gemein, dass ein Thread vor dem Eintritt in einen kritischen Abschnitt dessen Verfügbarkeit prüft und gegebenenfalls solange wartet, bis diese eintritt. Anschließend wird vermerkt, dass sich ein Thread im kritischen Abschnitt befindet. Beim Verlassen wird diese Markierung wieder entfernt. Bei einem Mutex wird die Zugangskontrolle über einen Wahrheitswert geregelt, weshalb hier maximal ein Thread den kritischen Bereich betreten kann.\footnote{vgl. \cite{oscon}, Abschnitt 5.5} Bei Semaphoren kommt ein Zähler zum Einsatz, über den die Anzahl der gleichzeitig erlaubten Threads gesteuert werden kann.\footnote{vgl. \cite{oscon}, Abschnitt 5.6} Hierbei ist sicher zu stellen, dass zwischen der Verfügbarkeitsprüfung und der Sperre für andere Threads kein anderer Thread diesen kritischen Abschnitt betreten kann.\footnote{Diese Operation muss nicht atomar im strengen Sinne sein. Kontextwechsel sind durchaus zu erlauben, jedoch nicht zu Threads, die um die betreffende Resource konkurrieren.} stattfindet. Code, der kritische Abschnitte korrekt behandelt wird auch als threadsicher bezeichnet.

Hierbei ist zu beachten, dass sich Verbraucherthreads und Erzeugerthreads nicht gegenseitig blockieren. Bei \ac{FIFO} Warteschlangen dürfen diese Operationen auch quasi gleichzeitig ablaufen. Hier können Verbraucher das älteste Element der Warteschlange entnehmen, während ein Erzeuger ein neues Element hinzufügt. Es bestehen somit zwei getrennte Kritische Abschnitte\footnote{„Prüfung ob Element vorhanden und Entnahme“ sowie „Prüfung ob Platz vorhanden und Einstellen”}, die unabhängig voneinander überwacht werden können.

Im Falle einer Realisierung mit Hilfe einer \ac{LIFO} Warteschlange besteht ein gemeinsamer kritischer Abschnitt. Erzeuger und Verbraucher können hier nicht gleichzeitig auf die Warteschlange zugreifen. Hierbei ist sicher zu stellen, dass die Akteure bei leerer bzw. voller Warteschlange den kritischen Abschnitt wieder verlassen, um den Programmfluss nicht dauerhaft in einem sogenannten „Deadlock“ zu blockieren.\footnote{Ein blockierender Verbraucher bei leerer Queue würde hierbei auch einen Erzeuger blockieren, der das erwartete Element somit nicht liefern kann.}

Ein weiterer kritischer Abschnitt der mit einem Mutex geschützt werden muss besteht bei der Ausgabe der in der Aufgabenstellung geforderten Informationen auf den Bildschirm. Hierbei könnte ein Thread von einem weiteren während der Ausgabe unterbrochen werden. Dieser zweite Thread würde dann seine Informationen zwischen die des Ersten schreiben. Um die Bildschirmausgabe konkurrieren alle Verbraucher und der Erzeuger.
% subsection kritische_abschnitte (end)

\subsection{Korrektes Logging} % (fold)
\label{sub:korrektes_logging}
Wie bereits in Abschnitt~\myref{sub:abgrenzungen} erwähnt, kann ein Akteure \code{A1} zwischen der Bildschirmausgabe und dem eigentlichen Zugriff auf die Warteschlange von einem anderen Akteur \code{A2} unterbrochen werden. Hierbei wird die Ausgabe von \code{A1} inkorrekt.

Dies kann dadurch vermieden werden, dass die Bildschirmausgabe ebenfalls in den kritischen Bereich mit aufgenommen wird. Hierzu ist statt den unabhängigen Locks für das Hinzufügen und das Entnehmen von Elementen ein Gemeinsames notwendig, welches einen Kontextwechsel zwischen Ausgabe und Zugriff verhindert. Hierdurch können sich bei ganz voller oder ganz leerer Warteschlange wartende Erzeuger und Verbraucher gegenseitig so blockieren, dass es zu einer Deadlocksituation kommt. Dies könnte vermieden werden, wenn Verbraucher bei leerer Warteschlange den kritischen Abschnitt direkt wieder verlassen, ohne ein Element zu entnehmen bzw. Erzeuger bei voller Warteschlange das aktuelle Element verwerfen, um den kritischen Abschnitt wieder verlassen zu können.

Eine weitere Möglichkeit, ist eine zweistufigen Ausgabe unter Nutzung eines Ticket--Systems: Ein Akteur betritt den kritischen Abschnitt des Ticket–Systems\footnote{Der kritische Abschnitt des Ticket–Systems ist für Erzeuger und Verbraucher gemeinsam.}, erzeugt die Ausgabezeile, speichert diese lokal und erhält ein „Ticket“ in Form einer fortlaufenden Nummer bevor er diesen ersten kritischen Abschnitt verlässt. Anschließend betritt der Akteur den kritischen Abschnitt der Warteschlange\footnote{Dieser ist wie im Abschnitt~\myref{sub:kritische_abschnitte} beschrieben für Erzeuger und Verbraucher unabhängig voneinander.} und prüft die Verfügbarkeit von Elementen bzw. Platz in der Warteschlange. Gegebenenfalls wird gewartet. Dann übermittelt der Akteur die Ticketnummer an das Ticketsystem. Hierdurch werden alle älteren Tickets ungültig. Ist das übergebene Ticket gültig, wird die Bildschirmausgabe durchgeführt, auf die Warteschlange zugegriffen und der kritische Abschnitt verlassen. Ist das übergebene Ticket ungültig, wird der kritische Abschnitt direkt verlassen, um den Vorgang wird ohne Wartezeit neu begonnen. 
% subsection korrektes_logging (end)

\subsection{Freiwilliges und unfreiwilliges Warten} % (fold)
\label{sub:freiwilliges_und_unfreiwilliges_warten}

Wie im Abschnitt~\myref{sub:klasse_akteur} noch näher beschrieben wird, besteht sowohl der Vorgang des Erzeugens als auch der des Verbrauchens im wesentlichen aus drei Phasen: Die Zeit, die für die Herstellung bzw. für den Konsum eines Elements benötigt wird\footnote{Dies schließt die Phase der „Sättigung“ mit ein.}, welche von außen als freiwillige Wartezeit wahrgenommen wird; die Wartezeit, die gegebenenfalls beim Warten auf den kritischen Abschnitt anfällt und somit als unfreiwillig anzusehen ist, sowie dem eigentlichen Zugriff auf die Warteschlange. Bei Untersuchungen konkreter Erzeuger–Verbraucher–Systeme können diese Zeiten entsprechend gemessen und unterschiedlich bewertet werden.

% subsection freiwilliges_und_unfreiwilliges_warten (end)

% section erzeuger_und_verbraucher (end)

\section{Genutzte Sprachmerkmale von Java} % (fold)
\label{sec:genutzte_sprachmerkmale_von_java}

\subsection{Collections, generische Klassen, Iteratoren} % (fold)
\label{sub:generics}
Die zu erstellende Anwendung benötigt Klassen, in denen mehrere Objekte\footnote{Hierbei handelt es sich um die Verbraucherthread–Objekte sowie um die Elemente der Warteschlange} gespeichert werden können. Die Java–Klassenbibliothek stellt hierzu verschiedene Containerklassen zur Verfügung, welche von der Klasse \code{Collection} abstammen\footnote{Ausnahme: \code{java.util.Map}, vgl. \cite{javaorange}, Seite 688} und Objekte beliebigen Typs aufnehmen.\footnote{vgl. \cite{javadoc:collection}}

Um dennoch eine Prüfung der Typen zu ermöglichen, wurde in Java ab der Version 5.0 das Konzept der generischen Programmierung (auch „parametrisierte Typen“) eingeführt. Hierbei kann bei der Instanziierung angegeben werden, welcher Klasse die in der Collection zu speichernden Objekte angehören werden. So wird durch die Deklaration \code{ArrayList <Verbraucher> v = new ArrayList<Verbraucher>();} ein Behälter vom Typ \code{ArrayList} erzeugt, der Elemente vom Typ \code{Verbraucher} speichert.\footnote{vgl. \cite{javaorange}, Seite 622ff}

Iteratoren stellen eine Möglichkeit dar, auf die in einer Collection gespeicherten Elemente zuzugreifen. Hierbei ist keine Kenntnis der tatsächlichen Stelle, an der das jeweilige Element gespeichert ist notwendig. Sind alle Elemente abgearbeitet, liefert die Methode \code{hasNext()} als Ergebnis \code{FALSE}.\footnote{vgl. \cite{javaorange}, Seite 692} In der zu erstellenden Anwendung wird ein Iterator implizit in Form einer erweiterten \code{for}–Schleife genutzt.
% subsection generics (end)

\subsection{Threads} % (fold)
\label{sub:die_klasse_thread}
Die im Abschnitt~\myref{sub:erzeuger_verbraucher_muster} beschriebene Unabhängigkeit und Gleichzeitigkeit der Akteure lässt sich in Java mit Hilfe von Threads implementieren. Hierbei handelt es sich um Anweisungsstränge die innerhalb eines Prozesses nebenläufig abgearbeitet werden.\footnote{vgl. \cite{javaorange}, Seite 748f}
Hierzu ist eine Ableitung der Klasse \code{Thread} zu erstellen, welche die Methode \code{run()} implementiert. Bei der Instanziierung der Objekte werden dann die entsprechenden Threads erzeugt, welche anschließend durch den Aufruf der Methode \code{start()} gestartet werden.\footnote{vgl. \cite{javaorange}, Seite 752f}

% subsection die_klasse_thread (end)

\subsection{Threadsicherheit} % (fold)
\label{sub:realisierung_mit_linkedblockingqueue}
Die Java–Klassenbibliothek bietet keine Semaphoren zur Sychronisierung von Threads an. Ein entsprechender Mechanismus müsste somit in einer eigenen Klasse implementiert werden. Hierbei ist besonders zu beachten, dass wie in Abschnitt~\myref{sub:kritische_abschnitte} beschrieben die Verfügbarkeitsprüfung und die Sperre der anderen Threads in einer quasi–atomaren Operation erfolgt. Dies kann mit Hilfe des Schlüsselwortes \code{synchronized} sichergestellt werden, welches eine \ac{Mutex} um den damit markierten Block bzw. um die damit ausgezeichnete Methode legt.\footnote{vgl. \cite{javaorange}, Seite 767} Alternativ können über die Klasse \code{ReentrantLock} von Codeblöcken unabhängige Sperren angefordert werden.\footnote{vgl. \cite{javadoc:rl}} Die Überprüfung der Queue auf vorhandene Elemente bzw. Platz zum Einfügen neuer Elemente muss hierbei genau so wie der eigentliche Zugriff sind hierbei ebenfalls zu kodieren.

Die generische Klasse \code{LinkedBlockingQueue<E>} implementiert alle im vorhergehenden Absatz genannten Aufgaben. Hierbei handelt es sich um eine FIFO–Queue, die threadsicheren Zugriff auf die Elemente ermöglicht. Beim Hinzufügen durch die Methode \code{put()} wird gegebenenfalls gewartet, bis Platz vorhanden ist, beim Entnehmen durch \code{take{}} wird gegebenenfalls gewartet, bis die Queue nicht mehr leer ist.\footnote{vgl. \cite{javadoc:lbq}} Intern verwendet die Klasse \code{ReentrantLock} zur Vermeidung von konkurrierenden Zugriffen\footnote{vgl. \cite{javadoc:lbqsource}, Zeile 377–402}

Die Methode \code{System.out.println()} verwendet intern \code{synchronized()} und ist somit threadsicher.\footnote{vgl. \cite{javadoc:println}, Zeile 771—776} Hierdurch ist auch der Zugriff auf die Bildschirmausgabe für den Fall geregelt, dass mehrere Threads gleichzeitig die Methode \code{log()} der Klasse \code{Logger} aufrufen.
% subsection realisierung_mit_linkedblockingqueue (end)

\subsection{Zufällige Wartezeiten} % (fold)
\label{sub:zufall}
Pseudo–Zufallszahlen werden in Java mithilfe von Objekten der Klasse Random generiert. Die Methode \code{nextInt(n)} liefert eine zufällige Ganzzahl r für die gilt $0 \leq r < n$. Durch die Addition von 1 kann hierdurch eine Wartezeit von mindestens einer Zeiteinheit erzeugt werden.

Gewartet wird innerhalb eines Thread mit der Methode \code{sleep()}, welche Wartezeit in Millisekunden als Argument erwartet. Daher kann die Zeiteinheit sehr klein gewählt werden, was einen schnellen Lauf der Simulation ermöglicht. Oder durch einen entsprechenden Faktor kann die Zeiteinheit größer gewählt werden, damit der Verlauf der Simulation besser live beobachtet werden kann.

% subsection zufall (end)

% section genutzte_sprachmerkmale_von_java (end)
