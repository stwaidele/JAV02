%!TEX root = /Users/stwaidele/Dropbox (Leisinger)/02 - AKAD/Projektbericht/Möglichkeiten der Digitalen Kontaktaufnahme im Endkundenbereich/vorlage.tex

\section{Erzeuger und Verbraucher} % (fold)
\label{sec:erzeuger_und_verbraucher}

\subsection{Erzeuger–Verbraucher—Muster} % (fold)
\label{sub:erzeuger_verbraucher_muster}

% subsection erzeuger_verbraucher_muster (end)

\subsection{Warteschlange} % (fold)
\label{sub:warteschlange}

\ac{LIFO} oder \textbf{\ac{FIFO}}, Heap, Stack oder \textbf{Ringbuffer}

% subsection warteschlange (end)

\subsection{Kritische Abschnitte} % (fold)
\label{sub:kritische_abschnitte}

Bei \ac{FIFO} \textit{müssten} gleichzeitiges Lesen und Schreiben möglich sein.

Erklärung von Mutex und Semaphoren.\footnote{vgl. \cite{oscon}}

% subsection kritische_abschnitte (end)

\subsection{Beachtenswerte Zustände} % (fold)
\label{sub:beachtenswerte_zustande}

Schreiben bei voller Queue, Lesen bei leerer Queue

% subsection beachtenswerte_zustande (end)

\subsection{Freiwilliges und unfreiwilliges Warten} % (fold)
\label{sub:freiwilliges_und_unfreiwilliges_warten}

% subsection freiwilliges_und_unfreiwilliges_warten (end)

\subsection{Realisierung mit Semaphoren und Mutex} % (fold)
\label{sub:realisierung_mit_semaphoren_und_mutex}

% subsection realisierung_mit_semaphoren_und_mutex (end)

\subsection{Realisierung mit LinkedBlockingQueue} % (fold)
\label{sub:realisierung_mit_linkedblockingqueue}

% subsection realisierung_mit_linkedblockingqueue (end)

% section erzeuger_und_verbraucher (end)

\newpage
\section{Genutzte Sprachmerkmale von JAVA} % (fold)
\label{sec:genutzte_sprachmerkmale_von_java}

\subsection{Generics} % (fold)
\label{sub:generics}

% subsection generics (end)

\subsection{Interfaces} % (fold)
\label{sub:interfaces}

% subsection interfaces (end)

\subsection{Die Klasse „Thread“} % (fold)
\label{sub:die_klasse_thread}

% subsection die_klasse_thread (end)

\subsection{Zufall} % (fold)
\label{sub:zufall}

% subsection zufall (end)

% section genutzte_sprachmerkmale_von_java (end)
