%!TEX root = /Users/stwaidele/Dropbox (Leisinger)/02 - AKAD/Projektbericht/Möglichkeiten der Digitalen Kontaktaufnahme im Endkundenbereich/vorlage.tex

\addtocontents{toc}{\protect\newpage}
\appendix
\addcontentsline{toc}{section}{Anhang} 


\section{Quelltext der Anwendung} % (fold)
\label{sec:quelltext_der_anwendung}

\subsection{Klasse: Pizzeria} % (fold)
\label{sub:klassepizzeria}
\begin{quellcode}[H]
\begin{footnotesize}
\begin{spacing}{1.0}
\inputminted[bgcolor=bg, tabsize=2, linenos]{java}{../JAV02-Pizzeria/src/Pizzeria.java}
\caption{Pizzeria.java}
\label{java:pizzeria}
\end{spacing}
\end{footnotesize}
\end{quellcode}
% subsection klasse_pizzeria (end)

\subsection{Klasse: Akteur} % (fold)
\label{sub:klasseakteur}
\begin{quellcode}[H]
\begin{footnotesize}
\begin{spacing}{1.0}
\inputminted[bgcolor=bg, tabsize=2, linenos]{java}{../JAV02-Pizzeria/src/Akteur.java}
\caption{Akteur.java}
\label{java:akteur}
\end{spacing}
\end{footnotesize}
\end{quellcode}
% subsection klasse_akteur (end)

\subsection{Klasse: Erzeuger} % (fold)
\label{sub:klasseerzeuger}
\begin{quellcode}[H]
\begin{footnotesize}
\begin{spacing}{1.0}
\inputminted[bgcolor=bg, tabsize=2, linenos]{java}{../JAV02-Pizzeria/src/Erzeuger.java}
\caption{Erzeuger.java}
\label{java:erzeuger}
\end{spacing}
\end{footnotesize}
\end{quellcode}
% subsection klasse_erzeuger (end)

\subsection{Klasse: Verbraucher} % (fold)
\label{sub:verbraucher}
\begin{quellcode}[H]
\begin{footnotesize}
\begin{spacing}{1.0}
\inputminted[bgcolor=bg, tabsize=2, linenos]{java}{../JAV02-Pizzeria/src/Verbraucher.java}
\caption{Verbraucher.java}
\label{java:verbraucher}
\end{spacing}
\end{footnotesize}
\end{quellcode}
% subsection verbraucher (end)

\subsection{Klasse: Queue} % (fold)
\label{sub:klassequeue}
\begin{quellcode}[H]
\begin{footnotesize}
\begin{spacing}{1.0}
\inputminted[bgcolor=bg, tabsize=2, linenos]{java}{../JAV02-Pizzeria/src/Queue.java}
\caption{Queue.java}
\label{java:queue}
\end{spacing}
\end{footnotesize}
\end{quellcode}
% subsection klasse_queue (end)

\subsection{Klasse: Logger} % (fold)
\label{sub:klasselogger}
\begin{quellcode}[H]
\begin{footnotesize}
\begin{spacing}{1.0}
\inputminted[bgcolor=bg, tabsize=2, linenos]{java}{../JAV02-Pizzeria/src/Logger.java}
\caption{Logger.java}
\label{java:logger}
\end{spacing}
\end{footnotesize}
\end{quellcode}
% subsection klasse_logger (end)
% section quelltext_der_anwendung (end)

\section{Ergebnisse verschiedenener Programmläufe} % (fold)
\label{sec:ergebnisse_verschiedenener_programmlaufe}

\subsection{Erzeuger schneller als Verbraucher} % (fold)
\label{sub:erzeuger_schneller_als_verbraucher}
\begin{quellcode}[H]
\begin{tiny}
\begin{spacing}{1.0}
\inputminted[bgcolor=bg, tabsize=2, linenos]{console}{esv.txt}
\caption{Erzeuger schneller als Verbraucher}
\label{out:esv}
\end{spacing}
\end{tiny}
\end{quellcode}

% subsection erzeuger_schneller_als_verbraucher (end)

% section ergebnisse_verschiedenener_programmlaufe (end)