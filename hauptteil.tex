\section{Implementierung} % (fold)
\label{sec:implementierung}

\subsection{Klasse: Akteur} % (fold)
\label{sub:klasse_akteur}

% subsection klasse_akteur (end)

\subsection{Klasse: Erzeuger} % (fold)
\label{sub:klasse_erzeuger}

% subsection klasse_erzeuger (end)

\subsection{Klasse: Verbraucher} % (fold)
\label{sub:klasse_verbraucher}

% subsection klasse_verbraucher (end)

\subsection{Klasse: Logger} % (fold)
\label{sub:klasse_logger}

% subsection klasse_logger (end)

\subsection{Sonstige Programmmerkmale} % (fold)
\label{sub:sonstige_programmmerkmale}

% subsection sonstige_programmmerkmale (end)
% section implementierung (end)

\newpage
\section{Laufzeitbetrachtungen} % (fold)
\label{sec:laufzeitbetrachtungen}

\subsection{Erzeuger schneller als Verbraucher} % (fold)
\label{sub:erzeuger_schneller_als_verbraucher}

% subsection erzeuger_schneller_als_verbraucher (end)

\subsection{Erzeuger langsamer als Verbraucher} % (fold)
\label{sub:erzeuger_langsamer_als_verbraucher}

% subsection erzeuger_langsamer_als_verbraucher (end)

\subsection{Ezeuger und Verbraucher gleich schnell} % (fold)
\label{sub:ezeuger_und_verbraucher_gleich_schnell}

% subsection ezeuger_und_verbraucher_gleich_schnell (end)

\subsection{Reduzierung auf einen Erzeuger und einen Verbraucher} % (fold)
\label{sub:reduzierung_auf_einen_erzeuger_und_einen_verbraucher}

Facade–Designpattern: Die n Verbraucher mit zufälligen Wartezeiten verhalten sich wie 1 Verbraucher mit kürzeren, aber ebenfalls zufälligen Wartezeiten. Dadurch ist die Anzahl der Verbraucher und analog dazu auch die der Erzeuger für die Betrachtung irrelevant.

% subsection reduzierung_auf_einen_erzeuger_und_einen_verbraucher (end)

% section laufzeitbetrachtungen (end)